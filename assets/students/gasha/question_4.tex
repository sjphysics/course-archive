\documentclass[9pt,addpoints]{exam}
\usepackage{enumitem}
\usepackage{amsfonts,amssymb,amsmath, amsthm}
\usepackage{graphicx}
\usepackage{systeme}
\usepackage{pgf,tikz,pgfplots}
\pgfplotsset{compat=1.15}
\usepgfplotslibrary{fillbetween}
\usepackage{mathrsfs}
\usetikzlibrary{arrows}
\usetikzlibrary{calc}
\pagestyle{headandfoot}
%\firstpageheadrule
\runningheadrule
\author{Gasha Abdisa}
\usepackage{geometry}
\geometry{
	a4paper,
	total={170mm,257mm},
	left=10mm,
	right=10mm,
	bottom=5mm,
	top=5mm,
}
\firstpagefooter{}{}{}
\runningfooter{}{}{}
\begin{document}
	\title{Assignment}
	\maketitle
	\textbf{Question 4 Solution}\\ \\
	\textbf{Part a} \\ \\
	For an electron with rest energy of 1 ev:
	$$\lambda=\dfrac{h}{p}=\dfrac{h}{\sqrt{2km}}$$
	$$\lambda=\dfrac{6.26\times10^{-34}Js}{\sqrt{2\times1ev\times511Kev/c^2}}$$
	$$\lambda=1.23nm$$
	\textbf{Part b} \\ \\
	We need to first find the momentum of the benzene molecule:
	$$p=mv=78g/mol\times1m/s=1.295\times10^{-25}kgm/s$$
	The de Broglie wave length is given by:
	$$\lambda=\dfrac{h}{p}=\dfrac{6.26\times10^{-34}Js}{1.295\times10^{-25}kgm/s}$$
	$$\lambda=0.512nm$$
	\textbf{Question 5 Solution}\\ \\
	We know that uncertainty principle is given by:
	$$\Delta x\times\Delta p=\dfrac{h}{4\pi}$$
	But, we also know, from de Broglie's law, that:
	$$\lambda=\dfrac{h}{p}\implies \dfrac{\Delta p}{p}=\dfrac{\Delta\lambda}{\lambda}$$
	$$\Delta p=p\dfrac{\Delta\lambda}{\lambda}=\dfrac{h}{\lambda}\dfrac{\Delta\lambda}{\lambda}=\dfrac{h\Delta\lambda}{\lambda^2}$$
	$$\Delta p=\dfrac{\Delta\lambda h}{\lambda^2}=\dfrac{10^{-6}\times6.626\times10^{-34}Js}{(10^{-10}m)^2}=6.626\times10^{-20}kgm/s$$
	Which means,
	$$\Delta x=\dfrac{h}{\Delta p\times4\pi}=\dfrac{6.62\times10^{-34}Js}{6.62\times10^{-20}kgm/s\times4\pi}$$
	$$\Delta x=\dfrac{10^{-14}}{4\pi}m$$
	\textbf{Question 10 Solution}\\ \\	
	\textbf{Part a} \\ \\
	The zero point energy is:
	$$E=h\omega(0+\frac{1}{2})=\dfrac{h\omega}{2}$$
	$\omega$ is given as follows:
	$$\omega=\sqrt{\dfrac{K}{\mu}}$$
	$$\omega=\sqrt{\dfrac{480.6Nm^{-1}}{6.053\times10^{-25}kg}}=\sqrt{7.94\times10^{24}/s^2}$$
	$$\omega=2.82\times10^{12}/s$$
	We can now calculate the energy since we have $\omega$
	$$E=\dfrac{h\omega}{2}$$
	$$E=\dfrac{6.626\times10^{-34}Js\times2.82\times10^{12}/s}{2}$$
	$$E=9.34\times10^{-22}J$$
	\textbf{Part b} \\ \\
	The zero point energy is:
	$$E=h\omega(0+\frac{1}{2})=\dfrac{h\omega}{2}$$
	$\omega$ is given as follows:
	$$\omega=\sqrt{\dfrac{K}{\mu}}$$
	$$\omega=\sqrt{\dfrac{480.6Nm^{-1}}{0.036458kg}}=\sqrt{13182.2/s^2}=114.8/s$$
	We can now calculate the energy since we have $\omega$
	$$E=\dfrac{h\omega}{2}$$
	$$E=\dfrac{6.626\times10^{-34}Js\times114.8/s}{2}$$
	$$E=\dfrac{7.61\times10^{-32}J}{2}$$
	$$E=3.80\times10^{-32}J$$
	\textbf{Question 11 Solution}\\ \\
	The transmission coefficient is given as follows:
	$$T\approxeq e^{-2bL}\text{ such that }b=\sqrt{\dfrac{8\pi^2m(U_b-E)}{h^2}}$$
	We first need to calculate b:
	$$b=\sqrt{\dfrac{8\pi^2m(U_b-E)}{h^2}}=\sqrt{\dfrac{8\pi^2\times9.11\times10^{-31}kg(5ev-1ev)}{(6.626\times10^{-34}Js)^2}}$$
	$$b=\sqrt{\dfrac{8\pi^2\times9.11\times10^{-31}kg(4ev)}{(6.626\times10^{-34}Js)^2}}$$
	$$b=\sqrt{1.05\times10^{20}/m^2}= 1.025\times10^{10}/m$$
	We can now find the transmission coefficient:
	$$T\approxeq e^{-2bL}$$
	$$T\approxeq e^{-2\times1.025\times10^{10}/m\times2\times10^{-9}m}$$
	$$T\approxeq e^{-40.99}$$
	$$T\approxeq 1.59\times10^{-18}$$
	\textbf{Question 12 Solution}\\ \\	
	
\end{document}