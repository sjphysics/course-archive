\documentclass[9pt,addpoints]{exam}
\usepackage{enumitem}
\usepackage{amsfonts,amssymb,amsmath, amsthm}
\usepackage{graphicx}
\usepackage{systeme}
\usepackage{pgf,tikz,pgfplots}
\pgfplotsset{compat=1.15}
\usepgfplotslibrary{fillbetween}
\usepackage{mathrsfs}
\usetikzlibrary{arrows}
\usetikzlibrary{calc}
\pagestyle{headandfoot}
%\firstpageheadrule
\runningheader{Homework 5}{}{Page \thepage\ of \numpages}
\runningheadrule
\author{Aaron GK}
\usepackage{geometry}
\geometry{
	a4paper,
	total={170mm,257mm},
	left=10mm,
	right=10mm,
	bottom=5mm,
	top=5mm,
}
\firstpagefooter{}{}{}
\runningfooter{}{}{}


\begin{document}
	\title{St John Baptist De La Salle Catholic School, Addis Ababa\\
		\large Homework 5 \\
		2nd Quarter}
	\maketitle
	\begin{center}
		\fbox{\fbox{\parbox{6in}{\centering
					Notes, and use of other aids is allowed.  Read all directions carefully and write your answers in the space provided.  To receive full credit, you must show all of your work. \textbf{Cheating or indications of cheating and similar answers will be punished accordingly}. 
		}}}
		\subsubsection*{Information}
		\begin{itemize}
			\item The homework is due on \textbf{Friday}, \textbf{December 30th}.
			\item You should Work on it \textbf{individually} and consult me if you have any questions. As I have reiterated multiple times, cheating will have a serious consequence.
			\item For purposes of neatness and simplicity of grading, you should do the homework on an \textbf{A-4 paper}.
		\end{itemize}
	\end{center}
	\begin{center}
		\subsection*{Questions}
	\end{center}
	
	\begin{questions}
		\question Define electric potential and describe its relationship with electric potential energy.
		\question When a 1.5 V flashlight battery runs a single 20W headlight, how many electrons pass through it each second?
		\question What is the difference between a car battery that is 12V and a smaller flashlight that is also 12V? Why does the latter run out faster although they have the same voltage?
		\question A lot of electrical appliances have potential differences set at some reference point. As you may recall, potential difference is the difference in absolute potential between points. For example, a voltaic cell might have a voltage of 1.5V, what does that mean? Also, why do we have ground as a reference in multiple electric appliances?
		\question How far apart are two conducting plates that have an electric field strength of  6.40×$10^3$V/m  between them, if their potential difference is 8.0kV?		
		\subsection*{Challenge Problems}
		\textit{\textbf{The following challenge problems are not required to be submitted, but are highly encouraged}.} \\
		\question Find the ratio of speeds of an electron and a negative hydrogen ion (one having an extra electron) accelerated through the same voltage.
		\question Why are equipotential lines and surfaces perpendicular to the electric field lines?
		\question A lightning bolt strikes a tree, moving 30.0 C of charge through a potential difference of 1.00×10$^2$MV.\begin{itemize}
			\item What energy was dissipated?
			\item  What mass of water could be raised from  room temperature(25$^0$c) to the boiling point and then boiled by this energy?
			\item Discuss the damage that could be caused to the tree by the expansion of the boiling steam.
		\end{itemize} 
	\end{questions}		
\end{document}