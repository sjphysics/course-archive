\documentclass[9pt,addpoints]{exam}
\usepackage{enumitem}
\usepackage{amsfonts,amssymb,amsmath, amsthm}
\usepackage{graphicx}
\usepackage{systeme}
\usepackage{pgf,tikz,pgfplots}
\pgfplotsset{compat=1.15}
\usepgfplotslibrary{fillbetween}
\usepackage{mathrsfs}
\usetikzlibrary{arrows}
\usetikzlibrary{calc}
\pagestyle{headandfoot}
%\firstpageheadrule
\runningheader{Homework 3}{}{Page \thepage\ of \numpages}
\runningheadrule
\author{Aaron GK}
\usepackage{geometry}
\geometry{
	a4paper,
	total={170mm,257mm},
	left=10mm,
	right=10mm,
	bottom=5mm,
	top=5mm,
}
\firstpagefooter{}{}{}
\runningfooter{}{}{}


\begin{document}
	\title{St John Baptist De La Salle Catholic School, Addis Ababa\\
		\large Homework 3 \\
		2nd Quarter}
	\maketitle
	\begin{center}
		\fbox{\fbox{\parbox{6in}{\centering
					Notes, and use of other aids is allowed.  Read all directions carefully and write your answers in the space provided.  To receive full credit, you must show all of your work. \textbf{Cheating or indications of cheating and similar answers will be punished accordingly}. 
		}}}
		\subsubsection*{Information}
		\begin{itemize}
			\item The homework is due on \textbf{Tuesday}, \textbf{December 13th}.
			\item You should Work on it \textbf{individually} and consult me if you have any questions. As I have reiterated multiple times, cheating will have a serious consequence.
			\item For purposes of neatness and simplicity of grading, you should do the homework on an \textbf{A-4 paper}.
		\end{itemize}
	\end{center}
	\begin{center}
		\subsection*{Questions}
	\end{center}
	
	\begin{questions}
		\question A grindstone with a mass of 50 kg and radius 1.5 m maintains a constant rotation rate of 8.0 rad/s by a motor while a knife is pressed against the edge with a force of 10.0 N. The coefficient of kinetic friction between the grindstone and the blade is 0.8. What is the power provided by the motor to keep the grindstone at the constant rotation rate?
		\question A potter’s disk spins from rest up to 8 rev/s in 10 s. The disk has a mass 5.0 kg and radius 50.0 cm. What is the angular momentum of the disk at  t=5s, what about t=10 s?(consider the moment of inertia of a disk to be $\text{I}=\dfrac{1}{2}\text{mR}^2$)
		\question A satellite of mass 2000 kg is in circular orbit about Earth. The radius of the orbit of the satellite is equal to two times the radius of Earth. (a) How far away is the satellite? (b) Find the kinetic, potential, and total energies of the satellite.
		\question The mass of a ball of radius 1.0 m is 6.0 kg. It rolls across a horizontal surface with a speed of 10.0 m/s. (a) How much work is required to stop the hoop? (b) If the hoop starts up a surface at  30°  to the horizontal with a speed of 10.0 m/s, how far along the incline will it travel before stopping and rolling back down? (\textit{Assume the ball is a hollow sphere, in which case its moment of inertia is given by:} $\text{I}=\dfrac{2}{5}\text{mr}^2)$
		\question A neutron star is a cold, collapsed star with nuclear density. A particular neutron star has a mass three times that of our Sun with a radius of 15.0 km. (a) What would be the weight of a 100-kg astronaut on standing on its surface? (b) What does this tell us about landing on a neutron star?
	\end{questions}		
\end{document}