\documentclass[9pt,addpoints]{exam}
\usepackage{enumitem}
\usepackage{amsfonts,amssymb,amsmath, amsthm}
\usepackage{graphicx}
\usepackage{systeme}
\usepackage{pgf,tikz,pgfplots}
\pgfplotsset{compat=1.15}
\usepgfplotslibrary{fillbetween}
\usepackage{mathrsfs}
\usetikzlibrary{arrows}
\usetikzlibrary{calc}
\pagestyle{headandfoot}
%\firstpageheadrule
\runningheader{Homework 4}{}{Page \thepage\ of \numpages}
\runningheadrule
\author{Aaron GK}
\usepackage{geometry}
\geometry{
	a4paper,
	total={170mm,257mm},
	left=10mm,
	right=10mm,
	bottom=5mm,
	top=5mm,
}
\firstpagefooter{}{}{}
\runningfooter{}{}{}


\begin{document}
	\title{St John Baptist De La Salle Catholic School, Addis Ababa\\
		\large Homework 4 \\
		2nd Quarter}
	\maketitle
	\begin{center}
		\fbox{\fbox{\parbox{6in}{\centering
					Notes, and use of other aids is allowed.  Read all directions carefully and write your answers in the space provided.  To receive full credit, you must show all of your work. \textbf{Cheating or indications of cheating and similar answers will be punished accordingly}. 
		}}}
		\subsubsection*{Information}
		\begin{itemize}
			\item The homework is due on \textbf{Monday}, \textbf{December 19th}.
			\item You should Work on it \textbf{individually} and consult me if you have any questions. As I have reiterated multiple times, cheating will have a serious consequence.
			\item For purposes of neatness and simplicity of grading, you should do the homework on an \textbf{A-4 paper}.
		\end{itemize}
	\end{center}
	\begin{center}
		\subsection*{Questions}
	\end{center}
	
	\begin{questions}
		\question Define electric field and factors affecting a field that emerges from a source charge.
		\question Discuss the differences between a source charge and a test charge.
		\question If there are two charges placed on the x-axis, a -2$\mu\text{C}$ charge at $\text{x}=3\text{nm}$ and a +40$\mu\text{C}$ charge at $\text{x}=-2\text{nm}$, compute the following:
		\begin{itemize}
			\item Calculate the electrostatic force between the two charges.
			\item If there is a charge of +1nC placed at the position $y=1.0\text{a}^0$, find the net force and electric field at that point.
		\end{itemize}
		\question Discuss the superposition principle and its applications.
		\question What is the electric field of the nucleus of a Hydrogen $^1_1\text{H}$ atom 1m away from the nucleus. Compare this to the acceleration due to gravity by the nucleus.
	\end{questions}
	\section*{Group Project}
	\begin{center}
		\subsubsection*{Information}
		\begin{itemize}
			\item The group project is due is due on \textbf{Monday}, \textbf{January 9th}.
			\item You should work on the projects \textbf{in groups} and consult me if you have any questions. Every member should have meaningful contribution to the project.
			\item The assignment should be submitted via email(\textbf{aaron@stjohn.edu.et} or \textbf{aaron@sjbdcs.org}) as a PDF file. You can alternatively choose to submit a hard-copy paper submission.
		\end{itemize}
	\end{center}
	\begin{center}
		\subsection*{Topics to work on}
	\end{center}
	\subsubsection*{Black Holes: history, origin \& scientific significance} 
	Groups 1 \& 7
	\subsubsection*{Mathematical Proof of Kepler's Laws}
	Groups 2 \& 8
	\subsubsection*{General Theory of Relativity}
	Groups 3 \& 9
	\subsubsection*{Application of Electrostatics}
	Groups 4 \& 10
	\subsubsection*{History of Electromagnetism \& Maxwell's Laws} 
	Groups 5 \& 6 \\ \\ 
	If there's an 11th group, they can choose which topic they like the most out of the ones given above and join them.		
	\section*{Midterm Prep}
		\begin{center}
		\fbox{\fbox{\parbox{6in}{\centering
					Notes, and use of other aids is \textbf{NOT} allowed.  Read all directions carefully and \textbf{write your answers in the answer sheet}.  To receive full credit, you must show all of your work.
		}}}
		\subsubsection*{Useful Constants}
		\begin{itemize}
			\item $\textbf{a}_g=10m/s^2\text{  - acceleration due to gravity}$\textbf{~}$\textbf{G} = 6.672\times10^{-12}\frac{Nm^2}{kg^2}\text{  - gravitational constant}$
		\end{itemize}
	\end{center}
	\begin{questions}
		\question If the net torque in a system is 0, what can we say about the angular velocity, angular momentum, the torques on the system? 
		\question What are the physics concent behind each of Kepler's Laws of Planetary Motion(try to prove each one)? 
		\question The trajectory of planets \& celestial bodies around the sun can only be a few shapes - what are they? 
		\question What are the factors affecting moment of inertia? How does the moment of inertia change if we change the factors?
		\question How was the gravitational constant discovered? 
		\question Why do we have the door handles away from the hinges?
		\question Prove Kepler's Third Law.(\textit{For the sake of simplicity, assume the path of planets is circular)}.
		\question If a planet is orbitting the sun 10 AU away, find its period.
		\question Convert the following quantities into their standard units:
		\begin{itemize}
			\item 2 rev/min
			\item 10 rev/s
			\item 900$^0$
			\item 100 revolution
			\item 7 rev/min$^2$
		\end{itemize}
	\end{questions}
	
\end{document}