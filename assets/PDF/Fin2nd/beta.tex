\documentclass[13pt,addpoints]{exam}
\usepackage{enumitem}
\usepackage{amsfonts,amssymb,amsmath, amsthm}
\usepackage{graphicx}
\usepackage{systeme}
\usepackage{pgf,tikz,pgfplots}
\pgfplotsset{compat=1.15}
\usepgfplotslibrary{fillbetween}
\usepackage{mathrsfs}
\usetikzlibrary{arrows}
\usetikzlibrary{calc}
\date{January, 2022}\usepackage{geometry}
\geometry{
	a4paper,
	total={170mm,257mm},
	left=10mm,
	right=10mm,
	bottom=10mm,
	top=5mm,
}
\pagestyle{headandfoot}
%\firstpageheadrule
\runningheader{Final Examination}{}{Page \thepage\ of \numpages}
\runningheadrule

\firstpagefooter{}{}{}
\runningfooter{By Aaron G.K. and Tariku D.}{}{Page \thepage\ of \numpages}


\begin{document}
	\title{St John Baptist De La Salle Catholic School, Addis Ababa\\
		\large Grade 10 Physics Final Examination \\
		4th Quarter}
	\maketitle
	\begin{center}
		\fbox{\fbox{\parbox{6in}{\centering
					Notes, and use of other aids is \textbf{NOT} allowed.  Read all directions carefully and write your answers in the space provided.  To receive full credit, you must show all of your work.}}}
		\subsubsection*{Useful Constants}
		\begin{itemize}
			\item $\textbf{e}=1.6\times10^{-19}\text{C}\text{  - elementary charge}$ \textbf{~}  $\textbf{m}_e=9.11\times10^{-31}\text{kg}\text{  - mass of an electron}$
			\item $\textbf{m}_p=1.673\times10^{-27}\text{kg}\text{  - mass of a proton}$ \textbf{~}$\mu_0=4\pi\times10^{-7}\frac{H}{m}\text{  - permeability of free space}$
			\item $\epsilon_0=8.85\times10^{-12}\frac{F}{m}\text{  - permitivitty of free space}$\textbf{~}$\textbf{G} = 6.672\times10^{-12}\frac{Nm^2}{kg^2}\text{  - gravitational constant}$
			\item $\textbf{N}_A = 6.022\times10^{23}\frac{1}{mol}\text{  - Avogadro's number}$$\textbf{~}$$\textbf{a}_g=10m/s^2\text{  - acceleration due to gravity}$
		\end{itemize}
	\end{center}
	{Name:\underline{\hspace{2in}}\text{     }{Roll Number:\underline{\hspace{0.5in}}\text{     }{Section:\underline{\hspace{0.3in}}{Time Allowed: \bf{1:45} hr}
				\subsubsection*{Multiple Choice Questions}
				\begin{questions}
					\question Which of the following changes to a parallel plate capacitor would not increase the energy stored in a capacitor at a fixed voltage?\\
					\begin{oneparchoices}
						\choice Increasing the area of the plates
						\choice Increasing the dielectric constant
						\choice Decreasing the charges on the plate
						\choice Increasing the distance between the plates
						\choice None of the above
					\end{oneparchoices}
					\question An electron enters a region of uniform electric field of 4$\times10^3$ V/m. What is the force on the electron?\\
					\begin{oneparchoices}
						\choice $8\times10^{-16}$ N
						\choice $6.4\times10^{-16}$ N
						\choice $8\times10^{16}$ N
						\choice $1.6\times10^{-19}$ N
						\choice None of the above
					\end{oneparchoices}
					\question Which of the following is true about gravitational potential energy?\\
					\begin{oneparchoices}
						\choice It is positive because gravity and mass are positive
						\choice It depends on the motion of the body
						\choice It depends on the height above the ground where the body is
						\choice It only depends on initial and final heights
						\choice None of the above
					\end{oneparchoices}
					\question What is the measure of the distribution of mass of a body in relation to its axis of rotation?\\
					\begin{oneparchoices}
						\choice Torque
						\choice Angular Momentum
						\choice Moment of Inertia
						\choice Linear Momentum
						\choice None of the above
					\end{oneparchoices}
					\question A point charge $Q_1$ is at $x=0$ and another point charge $Q_2$ is at $x=4$. What is the relationship between these two point charges if the absolute potential due to these charges is 0 at $x=8$?\\
					\begin{oneparchoices}
						\choice $Q_1=4Q_2$
						\choice $Q_1=-2Q_2$
						\choice $Q_1=-8Q_2$
						\choice $Q_1=4Q_2$
						\choice None of the above
					\end{oneparchoices}
					\question The potential difference between the terminals of a battery when the battery is isolated is:\\
					\begin{oneparchoices}
						\choice Electric Force
						\choice Terminal Voltage
						\choice Electromotive Force
						\choice Electrolytic Voltage
					\end{oneparchoices}
					\question A simple circuit consists of a load resistor of 10$\Omega$ connected to a battery of 18V EMF. If the current through the circuit is 1.6A, what is the internal resistance of the battery?\\
					\begin{oneparchoices}
						\choice 1.25$\Omega$
						\choice 12.5$\Omega$
						\choice 10$\Omega$
						\choice 1.6$\Omega$
						\choice None of the above
					\end{oneparchoices}
					\question The escape speed at the surface of some planet is twice that of the Earth's escape speed. What is the mass of the planet($M_p$) in terms of Earth's mass($M_e$)?\\
					\begin{oneparchoices}
						\choice $M_p$ = 0.5 $M_e$
						\choice $M_p$ = 2 $M_e$
						\choice $M_p$ = 4 $M_e$
						\choice $M_p$ = 8 $M_e$
						\choice None of the above
					\end{oneparchoices}
					\question One electron-volt is the same as:\\
					\begin{oneparchoices}
						\choice 3.6 J
						\choice 1.0 J
						\choice $3.6\times10^{6}$ J
						\choice $1.6\times10^{-19}$ J
					\end{oneparchoices}
					\question Electric companies usually list their billings in amounts of cents/KWh. For example, the British Power company EON bills consumers 7.3 cents/KWh. In this quantity, KWh is a unit of:\\
					\begin{oneparchoices}
						\choice Energy
						\choice Power
						\choice Current
						\choice Voltage
						\choice None of the above
					\end{oneparchoices}
					\question A 2$\mu$F and 1$\mu$F capacitors are connected in parallel and a potential difference is applied across the combination. The 2$\mu$F capacitor has:
					\begin{oneparchoices}
						\choice half the charge of the 1$\mu$F capacitor
						\choice twice the stored energy of the 1$\mu$F capacitor
						\choice twice the potential difference of the 1$\mu$F capacitor
						\choice half the stored energy of the 1$\mu$F capacitor
						\choice None of the above
					\end{oneparchoices}
					\question A tangent line to an equipotential surface and the electric field due to the same charge at any point must be:\\
					\begin{oneparchoices}
						\choice Parallel
						\choice Perpendicular
						\choice Opposite in direction
						\choice They don't have any relationship
						\choice None of the above
					\end{oneparchoices}
					\question Which of the following is true about resistivity and conductivity?\\
					\begin{oneparchoices}
						\choice They are reciprocals of one another
						\choice They are dimensionless quantities
						\choice They have direct relationship
						\choice They have the same SI units
						\choice None of the above
					\end{oneparchoices}
					\question If two, infinitely long parallel conducting wires carry the same current and the force per unit length on each wire is $2\times10^{-7}$ N/m, the current in each wire is defined to be:\\
					\begin{oneparchoices}
						\choice 1 Ampere
						\choice 1 Coulomb
						\choice $2\times10^{-7}$ Coulomb
						\choice $2\times10^{-7}$ Ampere
						\choice None of the above
					\end{oneparchoices}
					\question The angular impulse experienced by a body is equivalent to the change in:\\
					\begin{oneparchoices}
						\choice Mechanical energy
						\choice Linear Momentum
						\choice Angular Momentum
						\choice Relativistic Kinetic Energy
						\choice None of the above
					\end{oneparchoices}
					\question A 9V battery is connected to a 2$\mu$F capacitor. How much electric energy can be stored in the capacitor?\\
					\begin{oneparchoices}
						\choice 1.62$\times10^{-5}$ J
						\choice 8.1$\times10^{-5}$ J
						\choice 1.62$\times10^{-4}$ J
						\choice 8.1$\times10^{-4}$ J
						\choice None of the above
					\end{oneparchoices}
					\question The two ends of a 4$\Omega$ resistor are connected to a 16V battery. What is the total power delivered by the battery to the circuit?\\
					\begin{oneparchoices}
						\choice 4 W 
						\choice 16 W
						\choice 32 W
						\choice 64 W
						\choice None of the above
					\end{oneparchoices}
					\question Electric potential energy an energy of a charge possessed because it is in the:\\
					\begin{oneparchoices}
						\choice region of other masses
						\choice vacuum
						\choice region of zero electric field
						\choice region of other charges
						\choice None of the above
					\end{oneparchoices}
					\question Two resistors $R_1$ and $R_2$ are connected in series. If $R_1$=2$R_2$, which of the following is true?\\
					\begin{oneparchoices}
						\choice $V_1$=2$V_2$
						\choice $V_1$=$\dfrac{1}{2}$$V_2$
						\choice $I_1$=2$I_2$
						\choice $I_1$=$\dfrac{1}{2}$$I_2$
						\choice None of the above
					\end{oneparchoices}
					\question A 3A current is flowing through a Copper conductor ($n= 8.5\times10^{28}m^{-3}$) that has a cross sectional area of 1mm$^2$. What is the drift speed of the electrons in this conductor?\\
					\begin{oneparchoices}
						\choice $2.205\times10^{4}m/s$
						\choice $2.205\times10^{-4}m/s$
						\choice $2.205\times10^{-8}m/s$
						\choice $2.205\times10^{-2}m/s$
						\choice None of the above
					\end{oneparchoices}
					\question All conductors obey Ohm's Law.\\
					\begin{oneparchoices}
						\choice True
						\choice False
						\choice None of the above
					\end{oneparchoices}
					\question If the value of acceleration due to gravity on the surface of the Earth is \textbf{g}, what will its value be at a height equal to the radius of the Earth above the surface?\\ \\
					\begin{oneparchoices}
						\choice $\dfrac{\textbf{g}}{8}$
						\choice $\dfrac{\textbf{g}}{4}$
						\choice $\dfrac{\textbf{g}}{2}$
						\choice \textbf{g}
						\choice None of the above
					\end{oneparchoices}
					\question What is the potential at a distance of 10m from a charge 0f -5.0C?\\
					\begin{oneparchoices}
						\choice $-4.45\times10^{9}$ V
						\choice $+4.45\times10^{9}$ V
						\choice $-4.45\times10^{-9}$ V
						\choice $-4.45\times10^{+9}$ V
						\choice None of the above
					\end{oneparchoices}
					\question There are two parallel parallel charged plates in some region. A positive charge of 1.0$\times10^{-4}$ C is on the negatively charged plate. If the potential on the positively charged plate is +10KV and the potential on the negatively charged plate is -10KV, how much work is required to move the charge from the negative plate to the positive plate?\\
					\begin{oneparchoices}
						\choice 2.0 J
						\choice 0.0 J
						\choice 4.0 J
						\choice 1.0 J
						\choice None of the above
					\end{oneparchoices} 
					\question A charge of $Q_1=10\times10^{-9}$C is placed at the origin while another charge of $Q_2=10\times10^{-9}$C is placed at (0,6). What is the electric force on a third charge $Q_3=-2.5\times10^{-8}$C if it is placed at (4,3) due to $Q_1$ and $Q_2$?\\
					\begin{oneparchoices}
						\choice 1.08$\times10^{-7}$N, positive Y direction 
						\choice 1.42$\times10^{-7}$N, positive X direction
						\choice 1.42$\times10^{-7}$N, negative Y direction
						\choice 9.00$\times10^{-7}$N, positive Y direction
						\choice None of the above
					\end{oneparchoices}
					\subsubsection*{Conceptual \& Proof Problems}
					\question What are the factors affecting the resistance of conductor? List each factor and explain the effects of changing the factors on the resistance. \vspace{1in}
					\question Show that relation $R=\dfrac{\rho L}{A}$ follows from the macroscopic form of Ohm's Law($\implies V=IR$) and microscopic form($J=\sigma E$).\vspace{1in}
					\question Consider a region in space where a uniform electric field points in the positive Y direction. \begin{itemize}
						\item What is the orientation of the equipotential surfaces?\vspace{0.5in}
						\item If you move in the negative Y direction, does electric potential decrease or increase?\vspace{0.5in}
					\end{itemize}
					\question Why are equipotential lines and surfaces perpendicular to the electric field lines?\vspace{0.3in}
					\subsubsection*{Workout Problems}
					\question A capacitor in an RC circuit has a capacitance of 40$\mu$F while the resistor has a resistance of 20K$\Omega$. If the capacitor is initially empty, answer the following questions: ($Q(t)=Q(1-e^{-\dfrac{t}{\tau}})$)
					\begin{itemize}
						\item Calculate the amount of time it would take the charge in the capacitor to reach 63\%.\vspace{0.7in}
						\item Calculate the amount of charge left when $\dfrac{2}{5}\tau$ amount of time has dissipated.\vspace{0,7in}
					\end{itemize}
					\question How far apart are two conducting plates that have an electric field strength of  6.40×$10^3$V/m  between them, if one of the plates has a potential of -4.0KV and the other has a potential of 6.0KV?\vspace{1in}		
					\question On a planet whose radius is  \textbf{2R}, the acceleration due to gravity at the surface of the planet is \textbf{g/3}. What is the mass of the planet in terms of Earth's mass if the radius of the Earth is R and the acceleration due to gravity at the surface of the Earth is \textbf{g}?\vspace{1in} ($g=\dfrac{GM}{R^2}$)
					\subsubsection*{Extra Credit Problem}
					\question Calculate the Schwarzschild radius of a hypothetical subatomic particle that has a mass of 700Gev/c$^2$. Explain why you think it whether that this particle can ever turn into a black hole or not?
				\end{questions}
			\end{document}