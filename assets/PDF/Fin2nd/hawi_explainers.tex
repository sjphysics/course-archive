\documentclass[11pt,addpoints]{exam}
\usepackage{enumitem}
\usepackage{amsfonts,amssymb,amsmath, amsthm}
\usepackage{graphicx}
\usepackage{systeme}
\usepackage{pgf,tikz,pgfplots}
\pgfplotsset{compat=1.15}
\usepgfplotslibrary{fillbetween}
\usepackage{mathrsfs}
\usetikzlibrary{arrows}
\usetikzlibrary{calc}
\date{January, 2022}\usepackage{geometry}
\geometry{
	a4paper,
	total={170mm,257mm},
	left=10mm,
	right=10mm,
	bottom=10mm,
	top=5mm,
}
\pagestyle{headandfoot}
%\firstpageheadrule
\runningheader{Explainers}{}{Page \thepage\ of \numpages}
\runningheadrule

\firstpagefooter{}{}{}
\runningfooter{By Aaron G.K.}{}{Page \thepage\ of \numpages}


\begin{document}
	\title{Explanations for questions 1, 20 \& 22}
	\maketitle 
	
	Hi Hawi, here are the explanations for the questions you raised:
	\begin{questions}
					\question Which of the following changes to a parallel plate capacitor would not increase the energy stored in a capacitor at a fixed voltage?\\
					\begin{oneparchoices}
						\choice Increasing the area of the plates
						\choice Increasing the dielectric constant
						\choice Decreasing the charges on the plate
						\choice Increasing the distance between the plates
						\choice None of the above
					\end{oneparchoices}
					\\ \textbf{Answer:} C \\ \\
					This is the case because generally speaking, we know that $C=\dfrac{Q}{V}$ and under normal conditions(conditions in which the distance affects the voltage), we get capacitance as a function of area \& distance between plates:
					$$\text{V}=\text{Ed}$$
					$$\text{V}=\dfrac{\text{Q}}{\text{A}\varepsilon_0}\text{d}=\dfrac{\text{Qd}}{\text{A}\varepsilon_0}$$
					And from there, we can express it as follows:
					$$\text{C}=\dfrac{\text{Q}}{\text{V}}$$
					$$\text{C}=\dfrac{\text{Q}}{\dfrac{\text{Qd}}{\text{A}\varepsilon_0}}$$
					$$\text{C}=\dfrac{\varepsilon_0\text{A}}{\text{d}}$$
					But, you can see that all of this follows from the assumption that $V=Ed$ which works only when V depends on d. In this question, we have been told that voltage is constant regardless of whether E or d change. That brings us to a very important conclusion that d does not affect C because it doesn't affect the voltage.
					\question A 3A current is flowing through a Copper conductor ($n= 8.5\times10^{28}m^{-3}$) that has a cross sectional area of 1mm$^2$. What is the drift speed of the electrons in this conductor?\\
					\begin{oneparchoices}
						\choice $2.205\times10^{4}m/s$
						\choice $2.205\times10^{-4}m/s$
						\choice $2.205\times10^{-8}m/s$
						\choice $2.205\times10^{-2}m/s$
						\choice None of the above
					\end{oneparchoices} \textbf{Answer:} B \\ \\
					As you have shown in your paper $I=naev$ which means, $v=\dfrac{I}{nae}$
					$$v=\dfrac{I}{nae}=\dfrac{3A}{8.5\times10^{28}/m^3\times1.6\times10^{-19}C\times10^{-6}m^2}=\dfrac{3}{13.6\times10^{-3}}m/s=0.2205\times10^{-3}m/s$$
					$$v=0.2205\times10^{-3}m/s\times\dfrac{10}{10}=0.2205\times10\times\dfrac{10^{-3}}{10}m/s=2.205\times10^{-4}m/s$$
					\question If the value of acceleration due to gravity on the surface of the Earth is \textbf{g}, what will its value be at a height equal to the radius of the Earth above the surface?\\ \\
					\begin{oneparchoices}
						\choice $\dfrac{\textbf{g}}{8}$
						\choice $\dfrac{\textbf{g}}{4}$
						\choice $\dfrac{\textbf{g}}{2}$
						\choice \textbf{g}
						\choice None of the above
					\end{oneparchoices}
					\\ \textbf{Answer:} B \\ \\
					For this question, let's look back at Newton's Law of Universal Gravitation:
					$$F=\dfrac{GmM}{r^2}$$
					But, this force is the weight which means weight = $F$:
					$$mg=\dfrac{GmM}{r^2}$$
					$$g=\dfrac{GM}{r^2}$$
					So from, this we get the acceleration due to gravity as a function of distance from the larger mass. At the surface $\textbf{g}=\dfrac{GM}{R_E^2}$ but as we are at a height equal to the radius of the Earth, $r=2R_E$, which means:
					$$\textbf{g}_{new}=\dfrac{GM}{r^2}=\dfrac{GM}{(2R_E)^2}$$ 
					$$\textbf{g}_{new}=\dfrac{GM}{r^2}=\dfrac{GM}{4R_E^2}$$
					$$\textbf{g}_{new}=\dfrac{1}{4}\dfrac{GM}{R_E^2}=\dfrac{1}{4}\textbf{g}=\dfrac{\textbf{g}}{4}$$
				\end{questions}
			\end{document}