\documentclass[13pt,addpoints]{exam}
\usepackage{enumitem}
\usepackage{amsfonts,amssymb,amsmath, amsthm}
\usepackage{graphicx}
\usepackage{systeme}
\usepackage{pgf,tikz,pgfplots}
\pgfplotsset{compat=1.15}
\usepgfplotslibrary{fillbetween}
\usepackage{mathrsfs}
\usetikzlibrary{arrows}
\usetikzlibrary{calc}
\date{January, 2022}\usepackage{geometry}
\geometry{
	a4paper,
	total={170mm,257mm},
	left=10mm,
	right=10mm,
	bottom=10mm,
	top=5mm,
}
\pagestyle{headandfoot}
%\firstpageheadrule
\runningheader{Final Examination}{}{Page \thepage\ of \numpages}
\runningheadrule

\firstpagefooter{}{}{}
\runningfooter{By Aaron G.K. and Tariku D.}{}{Page \thepage\ of \numpages}


\begin{document}
	\title{St John Baptist De La Salle Catholic School, Addis Ababa\\
		\large Grade 10 Physics Final Examination \\
		2nd Quarter}
	\maketitle
	\begin{center}
		\fbox{\fbox{\parbox{6in}{\centering
					Notes, and use of other aids is \textbf{NOT} allowed.  Read all directions carefully and write your answers in the space provided.  To receive full credit, you must show all of your work.}}}
		\subsubsection*{Useful Constants}
		\begin{itemize}
			\item $\textbf{e}=1.6\times10^{-19}\text{C}\text{  - elementary charge}$ \textbf{~}  $\textbf{m}_e=9.11\times10^{-31}\text{kg}\text{  - mass of an electron}$
			\item $\textbf{m}_p=1.673\times10^{-27}\text{kg}\text{  - mass of a proton}$ \textbf{~}$\mu_0=4\pi\times10^{-7}\frac{H}{m}\text{  - permeability of free space}$
			\item $\epsilon_0=8.85\times10^{-12}\frac{F}{m}\text{  - permitivitty of free space}$\textbf{~}$\textbf{G} = 6.672\times10^{-12}\frac{Nm^2}{kg^2}\text{  - gravitational constant}$
			\item $\textbf{N}_A = 6.022\times10^{23}\frac{1}{mol}\text{  - Avogadro's number}$$\textbf{~}$$\textbf{a}_g=10m/s^2\text{  - acceleration due to gravity}$
		\end{itemize}
	\end{center}
	{Name:\underline{\hspace{2in}}\text{     }{Roll Number:\underline{\hspace{0.5in}}\text{     }{Section:\underline{\hspace{0.3in}}{Time Allowed: \bf{1:20} hr}
				\subsubsection*{Questions}
				\begin{questions}
					\question List two factors that affect the capacitance of capacitors and how.\vspace*{0.5in}
					\question An electron enters a region of uniform electric field of 5$\times10^2$ V/m. What is the force on the electron?\vspace*{0.4in}
					\question What is a conservative force?\vspace*{0.2in}
					\question What is moment of inertia?\vspace*{0.2in}
					\question A point charge $Q_1$ is at $x=0$ and another point charge $Q_2$ is at $x=2$. What is the relationship between these two point charges if the absolute potential due to these charges is 0 at $x=4$?\vspace*{0.5in}
					\question What is terminal voltage?\vspace*{0.3in}
					\question A simple circuit consists of a load resistor of 20$\Omega$ connected to a battery of 36V EMF. If the current through the circuit is 3.2A, what is the internal resistance of the battery?\vspace*{0.5in}
					\question How much is one electron-volt in Joules?\vspace*{0.5in}
					\question Electric companies usually list their billings in amounts of cents/KWh. In this quantity, what is KWh a unit of:\vspace*{0.1in}
					\subsubsection*{Multiple Choice Questions}
					\question A 2$\mu$F and 1$\mu$F capacitors are connected in parallel and a potential difference is applied across the combination. The 2$\mu$F capacitor has:
					\begin{oneparchoices}
						\choice half the charge of the 1$\mu$F capacitor
						\choice twice the stored energy of the 1$\mu$F capacitor
						\choice twice the potential difference of the 1$\mu$F capacitor
						\choice half the stored energy of the 1$\mu$F capacitor
						\choice None of the above
					\end{oneparchoices}
					\question A tangent line to an equipotential surface and the electric field due to the same charge at any point must be:\\
					\begin{oneparchoices}
						\choice Parallel
						\choice Perpendicular
						\choice Opposite in direction
						\choice They don't have any relationship
						\choice None of the above
					\end{oneparchoices}
					\question Which of the following is true about resistivity and conductivity?\\
					\begin{oneparchoices}
						\choice They are reciprocals of one another
						\choice They are dimensionless quantities
						\choice They have direct relationship
						\choice They have the same SI units
						\choice None of the above
					\end{oneparchoices}
					\question If two, infinitely long parallel conducting wires carry the same current and the force per unit length on each wire is $2\times10^{-7}$ N/m, the current in each wire is defined to be:\\
					\begin{oneparchoices}
						\choice 1 Ampere
						\choice 1 Coulomb
						\choice $2\times10^{-7}$ Coulomb
						\choice $2\times10^{-7}$ Ampere
						\choice None of the above
					\end{oneparchoices}
					\question The angular impulse experienced by a body is equivalent to the change in:\\
					\begin{oneparchoices}
						\choice Mechanical energy
						\choice Linear Momentum
						\choice Angular Momentum
						\choice Relativistic Kinetic Energy
						\choice None of the above
					\end{oneparchoices}
					\question A 18V battery is connected to a 2$\mu$F capacitor. How much electric energy can be stored in the capacitor?\\
					\begin{oneparchoices}
						\choice 1.62$\times10^{-5}$ J
						\choice 8.1$\times10^{-5}$ J
						\choice 1.62$\times10^{-4}$ J
						\choice 8.1$\times10^{-4}$ J
						\choice None of the above
					\end{oneparchoices}
					\question The two ends of a 4$\Omega$ resistor are connected to a 8V battery. What is the total power delivered by the battery to the circuit?\\
					\begin{oneparchoices}
						\choice 4 W 
						\choice 16 W
						\choice 32 W
						\choice 64 W
						\choice None of the above
					\end{oneparchoices}
					\question Gravitational potential energy an energy of a charge possessed because it is in the:\\
					\begin{oneparchoices}
						\choice region of other masses
						\choice vacuum
						\choice region of zero electric field
						\choice region of other charges
						\choice None of the above
					\end{oneparchoices}
					\question Two resistors $R_1$ and $R_2$ are connected in series. If $R_1$=2$R_2$, which of the following is true?\\
					\begin{oneparchoices}
						\choice $V_1$=2$V_2$
						\choice $V_1$=$\dfrac{1}{2}$$V_2$
						\choice $I_1$=2$I_2$
						\choice $I_1$=$\dfrac{1}{2}$$I_2$
						\choice None of the above
					\end{oneparchoices}
					\question A 6A current is flowing through a Copper conductor ($n= 8.5\times10^{28}m^{-3}$) that has a cross sectional area of 1mm$^2$. What is the drift speed of the electrons in this conductor?\\
					\begin{oneparchoices}
						\choice $4.41\times10^{4}m/s$
						\choice $4.41\times10^{-4}m/s$
						\choice $4.41\times10^{-8}m/s$
						\choice $4.41\times10^{-2}m/s$
						\choice None of the above
					\end{oneparchoices}
					\question All conductors obey Ohm's Law.\\
					\begin{oneparchoices}
						\choice True
						\choice False
						\choice None of the above
					\end{oneparchoices}
					\question What is the potential at a distance of 10m from a charge 0f 5.0C?\\
					\begin{oneparchoices}
						\choice $-4.45\times10^{-9}$ V
						\choice $+4.45\times10^{9}$ V
						\choice $-4.45\times10^{-9}$ V
						\choice $-4.45\times10^{+9}$ V
						\choice None of the above
					\end{oneparchoices}
					\question A charge of $Q_1=10\times10^{-9}$C is placed at the origin while another charge of $Q_2=10\times10^{-9}$C is placed at (0,6). What is the electric force on a third charge $Q_3=-2.5\times10^{-8}$C if it is placed at (4,3) due to $Q_1$ and $Q_2$?\\
					\begin{oneparchoices}
						\choice 1.08$\times10^{-7}$N, positive Y direction 
						\choice 1.42$\times10^{-7}$N, positive X direction
						\choice 1.42$\times10^{-7}$N, negative Y direction
						\choice 9.00$\times10^{-7}$N, positive Y direction
						\choice None of the above
					\end{oneparchoices}
					\question There are two parallel parallel charged plates in some region. A positive charge of 1.0$\times10^{-4}$ C is on the negatively charged plate. If the potential on the positively charged plate is +10KV and the potential on the negatively charged plate is -10KV, how much work is required to move the charge from the negative plate to the positive plate?\\
					\begin{oneparchoices}
						\choice 2.0 J
						\choice 0.0 J
						\choice 4.0 J
						\choice 1.0 J
						\choice None of the above
					\end{oneparchoices} 
					\question If the value of acceleration due to gravity on the surface of the Earth is \textbf{g}, what will its value be at a height equal to twice the radius of the Earth above the surface?\\ \\
					\begin{oneparchoices}
						\choice $\dfrac{\textbf{g}}{9}$
						\choice $\dfrac{\textbf{g}}{4}$
						\choice $\dfrac{\textbf{g}}{2}$
						\choice \textbf{g}
						\choice None of the above
					\end{oneparchoices}
					\subsubsection*{Conceptual \& Proof Problems}
					\question What are the factors affecting the capacitance of a parallel plate capacitor? List each factor and explain the effects of changing the factors on the capacitance. \vspace{1in}
					\question Show that the capacitance of a parallel plate capacitor is given by $C=\dfrac{\varepsilon A}{d}$\vspace{1in}
					\question Consider a region in space where a uniform electric field points in the positive Z direction. \begin{itemize}
						\item What is the orientation of the equipotential surfaces?\vspace{0.5in}
						\item If you move in the negative Y direction, does electric potential decrease or increase? Why?\vspace{0.5in}
					\end{itemize}
					\question Show that equipotential lines and surfaces are perpendicular to the electric field lines.\vspace{0.3in}
					\subsubsection*{Workout Problems}
					\question How far apart are two conducting plates that have an electric field strength of  6.40×$10^3$V/m  between them, if one of the plates has a potential of 2.0KV and the other has a potential of 12.0KV?\vspace{1in}
					\question On a planet whose radius is  \textbf{8R}, the acceleration due to gravity at the surface of the planet is \textbf{g/2}. What is the mass of the planet in terms of Earth's mass if the radius of the Earth is R and the acceleration due to gravity at the surface of the Earth is \textbf{g}?\vspace{1in} ($g=\dfrac{GM}{R^2}$)
					\question A capacitor in an RC circuit has a capacitance of 40$\mu$F while the resistor has a resistance of 20K$\Omega$. If the capacitor is initially full, answer the following questions: ($Q(t)=Q(1-e^{-\dfrac{t}{\tau}})$)
					\begin{itemize}
						\item Calculate the amount of time it would take the charge in the capacitor to drop to 37\%.\vspace{0.7in}
						\item Calculate the amount of charge left when $\dfrac{3}{5}\tau$ amount of time has dissipated.\vspace{0,7in}
					\end{itemize}
					\question Show that for an RC circuit, the voltage as a function of time while the capacitor is charging is given by:
					$$V(t)=V(1-e^{-t/\tau})$$
				\end{questions}
			\end{document}