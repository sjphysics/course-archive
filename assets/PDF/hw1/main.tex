\documentclass[9pt,addpoints]{exam}
\usepackage{enumitem}
\usepackage{amsfonts,amssymb,amsmath, amsthm}
\usepackage{graphicx}
\usepackage{systeme}
\usepackage{pgf,tikz,pgfplots}
\pgfplotsset{compat=1.15}
\usepgfplotslibrary{fillbetween}
\usepackage{mathrsfs}
\usetikzlibrary{arrows}
\usetikzlibrary{calc}
\pagestyle{headandfoot}
%\firstpageheadrule
\runningheader{Homework 1}{}{Page \thepage\ of \numpages}
\runningheadrule
\author{Aaron GK}
\usepackage{geometry}
\geometry{
	a4paper,
	total={170mm,257mm},
	left=10mm,
	right=10mm,
	bottom=5mm,
	top=5mm,
}
\firstpagefooter{}{}{}
\runningfooter{}{}{}


\begin{document}
	\title{St John Baptist De La Salle Catholic School, Addis Ababa\\
		\large Homework 1 \\
		2nd Quarter}
	\maketitle
	\begin{center}
		\fbox{\fbox{\parbox{6in}{\centering
					Notes, and use of other aids is allowed.  Read all directions carefully and write your answers in the space provided.  To receive full credit, you must show all of your work. \textbf{Cheating or indications of cheating and similar answers will be punished accordingly}. 
		}}}
		\subsubsection*{Information}
		\begin{itemize}
			\item The homework is due on \textbf{Monday}, \textbf{November 28th}.
			\item You should Work on it \textbf{individually} and consult me if you have any questions. As I have reiterated multiple times, cheating will have a serious consequence.
			\item For purposes of neatness and simplicity of grading, you should do the homework on an \textbf{A-4 paper}.
		\end{itemize}
	\end{center}
\begin{center}
		\subsection*{Questions}
\end{center}
	
				
				\begin{questions}
					\question We have seen the equations of uniformly accelerated motion(\textbf{SUVAT equations}) and used them multiple times. Recall that for rotational motion, the kinematics are very similar to its linear counterparts. For this question, derive the equations we use when the angular acceleration is constant($\varDelta\alpha=0$).
					\question Show the relationships between linear and angular physical quantities and derive how they relate to one another.
					\question When the equation of centripetal acceleration is derived, we find that it is given by:
					$$\textbf{a}_c=-\dfrac{v^2}{r}(\cos(\omega t)\hat{\textbf{i}}+\sin(\omega t)\hat{\textbf{j}})$$
					\begin{itemize}
						\item Discuss what the negative sign is indicative of.
						\item How does the $(\cos(\omega t)\hat{\textbf{i}}+\sin(\omega t)\hat{\textbf{j}})$ term not affect the magnitude of the centripetal acceleration?
					\end{itemize}
					\question For an object in a circular motion, we have two accelerations; the tangential acceleration and the centripetal acceleration. Discuss the differences between the two and derive an equation for the resultant acceleration using only \textit{r}, \textit{$\omega$}, and \textit{$\alpha$}.
					\question A disk is rotating in a uniform circular motion with an axis of rotation that is out of the page. Answer the following questions:
						\begin{itemize}
							\item What direction is the disk rotating in?
							\item We put two rocks, a \& b on the disk at $\dfrac{r}{3}$ and $\dfrac{2r}{5}$ distances from the axis of rotation. Which of the two rocks has a larger \textit{linear} speed?
						\end{itemize}
					\question Answer the following questions:
					\begin{itemize}
						\item What is the period of rotation of Earth in seconds? 
						\item What is the angular velocity of Earth?
						\item Given that Earth has a radius of  $6.4\times10^6$m at its equator, what is the linear velocity at Earth's surface?
				\end{itemize}
					\question A motor engine has a radius of 16 cm and rotates at 8000 rev/min at its maximum performance.
					\begin{itemize}
						\item Calculate the magnitude of the centripetal acceleration at its edge in meters per second squared and convert it to multiples of \textbf{\textit{g}}.
						\item What is the linear speed of a point on its edge?
					\end{itemize}
					
				\end{questions}		
			\end{document}