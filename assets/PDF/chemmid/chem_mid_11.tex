\documentclass[12pt,addpoints]{exam}
%\usepackage{enumitem}
\usepackage{amsfonts,amssymb,amsmath, amsthm}
\usepackage{graphicx}
\usepackage{systeme}
\usepackage{pgf,tikz,pgfplots}
\pgfplotsset{compat=1.15}
\usepgfplotslibrary{fillbetween}
\usepackage{mathrsfs}
\usetikzlibrary{arrows}
\usetikzlibrary{calc}
\usepackage{geometry}
\geometry{
	a4paper,
	total={170mm,257mm},
	left=20mm,
	top=20mm,
}
\date{March, 2023}
\pagestyle{headandfoot}
%\firstpageheadrule
\runningheader{3rd Quarter Midterm Examination}{}{Page \thepage\ of \numpages}
\runningheadrule
\firstpagefooter{}{}{}
\runningfooter{By Dagmawi Tsegaw}{}{Page \thepage\ of \numpages}
\begin{document}
	\title{St John Baptist De La Salle Catholic School, Addis Ababa\\
		\large Grade 11 Chemistry Midterm Examination \\
		$3^\text{rd}$ Quarter}
	\maketitle
	{Name:\underline{\hspace{2in}}\text{     }{Roll Number:\underline{\hspace{0.5in}}\text{     }{Section:\underline{\hspace{0.3in}}{Time Allowed: \bf{45  minutes}}
				\subsubsection*{Multiple Choice Questions}
				\begin{questions}
					\question Which of the following is characteristics of an antibonding molecular orbital?
					\begin{choices}
						\choice It is a molecular orbital with a high probability of finding electrons parallel to the region between the bonded atoms.
						\choice It has no electrons
						\choice It is a molecular orbital with a high probability of finding electrons away from the region between bonded atoms.
						\choice It is a molecular orbital with a high probability of of finding electrons in the region between bonded atoms.
					\end{choices}
					\question Identify the statement which is correct about the $O_2$ molecule and the $O_2^{2-}$ ion.
					\begin{choices}
						\choice The bond order of $O_2^{2-}$ is less than that of $O_2$.
						\choice $O_2$ is diamagnetic whereas $O_2^{2-}$ is paramagnetic.
						\choice The oxygen-oxygen bond is stronger in $O_2^{2-}$.
						\choice The number of electrons in anti-bonding orbital is less in $O_2^{2-}$.
					\end{choices}
					\question Which one of the following represents the correct electron pair geometry and molecular geometry of $Cl_2O$?
					\begin{choices}
						\choice Linear, Linear
						\choice Tetrahedral, bent
						\choice Trigonal planar, T-shaped
						\choice Square planar, linear
					\end{choices}
					\question Which one of the following molecules is non-polar? \\
					\begin{oneparchoices}
						\choice $CS_2$
						\choice $SO_2$
						\choice $CHCl_3$
						\choice $SF_4$
					\end{oneparchoices}
					\question Which of the following molecules has the shortest bond length? \\
					\begin{oneparchoices}
						\choice $O_2$
						\choice $Cl_2$
						\choice $N_2$
						\choice $Br_2$
					\end{oneparchoices}
					\question There is a strong covalent bond between the N atoms in nitrogen gas, $N_2$, why, then, does nitrogen have such a low boiling point of -196$^0$C?
					\begin{choices}
						\choice The bond between the N-atoms is triple.
						\choice N is very electronegative, only next to F and O.
						\choice The strong bond, at intramolecular one determines the boiling point of the substance.
						\choice Boiling point is determined by intermolecular force, which in this case is weak as the molecular is non-polar.
					\end{choices} 
					\question What are intermolecular forces? They are forces due to the attraction between 
					\begin{choices}
						\choice cations and anions.
						\choice molecules.
						\choice cations and delocalized electrons.
						\choice nuclei and electron pair.
					\end{choices}
					\question Which of the following belongs to chemical bonding theories?
					\begin{choices}
						\choice Valence bond theory and molecular orbital theory.
						\choice Kinetic-molecular theory and valence shell electron pair repulsion theory.
						\choice Valence bond theory and valence shell electron pair repulsion theory.
						\choice Molecular orbital theory and kinetic-molecular theory.
					\end{choices}
					\question The hybridization of the central atom in the $XeF_4$ molecule is \\
				 	\begin{oneparchoices}
				 		\choice $sp^2$
				 		\choice $sp^3$
				 		\choice $sp^3d$
				 		\choice $sp^3d^2$
				 	\end{oneparchoices}
			 		\question The dissolution of water in octane($C_8H_{18}$) is prevented by
			 		\begin{choices}
			 			\choice dipole-dipole attraction between octane molecules.
			 			\choice hydrogen bonding between water molecules.
			 			\choice London dispersion forces between octane molecules.
			 			\choice repulsion between like-charged water and octane molecules.
			 		\end{choices}
		 			\question Given the following $AF_n$ species, $BF_3$, $BeF_2$, $CF_4$, $NF_3$, $OF_2$, what is the correct order of the F-A-F bond angles?
		 			\begin{choices}
		 				\choice $OF_2<BeF_2<NF_3<BF_3<CF_4$
		 				\choice $BeF_2<OF_2<NF_3<BF_3<CF_4$
		 				\choice $CF_4<BF_3<NF_3<BeF_2<OF_2$
		 				\choice $OF_2<NF_3<CF_4<BF_3<BeF_2$
		 			\end{choices}
	 				\question From $CO_2$, $H_2O$, $BeCl_2$, and $N_2O$ which have the same molecular geometry?
	 				\begin{choices}
	 					\choice $CO_2$, $H_2O$, and $N_2O$
	 					\choice $CO_2$, $BeCl_2$, and $N_2O$
	 					\choice $CO_2$ and $BeCl_2$ only
	 					\choice $H_2O$ and $N_2O$ only
	 				\end{choices}
 					\question The number of resonance structures for $CO_3^{2-}$ are: \\
 					\begin{oneparchoices}
 						\choice 9
 						\choice 2
 						\choice 3
 						\choice 6
 					\end{oneparchoices}
 					\question In the following equation, what type of hybridization change, if any occurs at $Xe$ atom?
 					$$XeF_2(S)+F_2(g)\rightarrow XeF_4(s)$$
 					\begin{oneparchoices}
 						\choice $sp^3d$ to $sp^3d^2$
 						\choice $dsp^2$ to $sp^3$
 						\choice $sp^3$ to $sp^3d$
 						\choice $sp^3d$ to $sp^3$
 					\end{oneparchoices}
 					\question Which of the following is correct about the type of overlaps that describe the triple bonds in nitrogen molecule in which the orbitals of the two nitrogen are designated with the subscripts 1 and 2?
 					\begin{choices}
 						\choice $sp^2_1-----sp^2_2$ and $p_{x1}-p_{y1}$ and $p_{x2}-p_{y2}$
 						\choice $sp_1-----sp_2$ and $p_{x1}-p_{y2}$ and $p_{x2}-p_{y1}$
 						\choice $sp^3_1-----sp^3_2$ and $p_{x1}-p_{y1}$ and $p_{x2}-p_{y2}$
 						\choice $sp^2_1-----sp^2_2$ and $p_{x1}-p_{y2}$ and $p_{x2}-p_{y1}$
 					\end{choices}
 					\question The dipole moment is the highest for: \\
 					\begin{oneparchoices}
 						\choice Trans-2-butene 
 						\choice 1,3 - dimethyl benzene
 						\choice Acetophenone 
 						\choice Ethanol
 					\end{oneparchoices}
 					\question Which electron is most electronegative? \\
 					\begin{oneparchoices}
 						\choice $Sp^3$ hybridized
 						\choice $Sp^2$ hybridized
 						\choice $Sp$ hybridized
 						\choice $Sp^3d$ hybridized
 					\end{oneparchoices}
 					\subsection*{Workout}
 					\question Write the molecular orbital diagrams for the carbide ion($C_2^{2-}$) \vspace{2in}
 					\question A neutral molecule having the general formula $AB_3$ has two unshared pair of electrons on A. What is the hybridization of A? \vspace{1.5in}
 					\question What is the bond order of $O_2^{+}$\vspace{1in}
				\end{questions}
			\end{document}