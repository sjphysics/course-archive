\documentclass[11pt]{article}
%\voffset -.5in \hoffset -.5in \textheight 9.9in \textwidth 6.4in
\usepackage{geometry,url}
\usepackage{hyperref}
\geometry{body={7in,9.3in}, centering}
\pagestyle{empty}
\usepackage{multicol}
\begin{document}
	\begin{center}
		\textbf{\Large 3rd Quarter Grade 10 Physics}\\
		St John Baptist De La Salle Catholic School, Addis Ababa\\ 
	\end{center}
	%\framebox[6.2in]{\makebox[\totalheight]}
	
	%\begin{tabular}{l@{\qquad}l@{\qquad}l@{\qquad}}
	%   Course Title: & &  \\
	%  \end{tabular}
------------------------------------------------------------------------------------------------------------------------------------
\vspace{-.1in}
\begin{center}
	\begin{tabular}{llll}
		\textbf{Instructor:} & \footnotesize{Aaron GK}  &   \textbf{Email:} & \footnotesize{aaron@sjbdcs.org} \\ 
		\textbf{Office Hours:} & \footnotesize{7:30am-8:30am,3:30pm-5pm}   & \textbf{Term:}  & \footnotesize{3rd Quarter}  \\
		\textbf{Course Site:} & \footnotesize{physics.kebede.org}   & \textbf{Credit Hours:} & 4\\
	\end{tabular}
\end{center}

------------------------------------------------------------------------------------------------------------------------------------
%{Spring 2011}

\noindent {\bf COURSE DESCRIPTION}\\
Indefinite and definite integrals, probability, functions of several variables, least squares, differential equations.

\medskip\noindent {\bf STUDENT LEARNING OUTCOMES}\\
The successful student, at the end of the course, should be able to
\begin{itemize}
	\item\textbf{\emph{Vector Algebra}}:  Understand and manipulate vectors algebraically while adding and multiplying vectors\vspace{-.1in}
	\item\textbf{\emph{Magnetic Fields \& Force}}: Understand the cause of magnetic fields, plot the fields caused by various sources, calculate the field by various sources, calculate the force using vector products and the right hand rule.\vspace{-.1in}
	\item\textbf{\emph{Magnetic Force on a Current Carrying Wires \& Charges}}:  Calculate the magnetic force(and torque) caused by a magnetic field on a current carrying wire or a moving charge. Work out the directions using vector products.\vspace{-.1in}
	\item\textbf{\emph{Electromagnetic Induction}}:  Understand how we can induce current from magnetic field, calculate the amount of induced current and voltage. Look at the applications of electromagnetic induction and how it applies in our everyday lives.\vspace{-.1in}
	\item\textbf{\emph{Inductance}}:  Understand what inductance is, what causes it and the types of inductance using Faraday's and Lenz's laws.
\end{itemize}

\noindent{\bf TEXTBOOK:} We use three books with the MOE Old text book as a base. The other two books are the new MOE text book and OpenStax AP Physics Textbook
\begin{itemize}
	\item Base Text: \url{https://sj.kebede.org/Text%20Books/Physics%20Student%20G10.pdf}
	\item New MOE Textbook: \url{https://physics.kebede.org/assets/PDF/textbook.pdf}
	\item OpenStax AP Physics: \url{https://openstax.org/details/books/college-physics-ap-courses}
\end{itemize}
\noindent{\bf Important Dates:}

\newpage
\medskip\noindent{\bf CLASS SCHEDULE}

\halign{#\hfill&\qquad#\hfill&\qquad#\hfill&\qquad#\hfill&\qquad#\hfill\cr Chapter& Section& Topic/Learning Outcome& Hours\cr \noalign{\bigskip}
	
	Chapter & 5 & Applications of Differentiation   & (2 hours) \cr
	
	& 5.8 & Antiderivatives;  \textbf{\emph{Indefinite integrals}} & 2.0 \cr
	
	Chapter & 6 & Integration   & (7 hours) \cr
	
	& 6.1 & The Definite Integral; \textbf{\emph{Definite integrals}} & 4.0\cr
	
	& 6.2 & The Fundamental Theorem of Calculus; \textbf{\emph{Definite integrals}} & 1.0 \cr
	
	& 6.3 & Applications of Integration; \textbf{\emph{Definite integrals}}& 2.0 \cr
	
	Chapter & 7 & Integration Techniques and Computational Methods   & (7 hours) \cr
	
	& 7.1 & The Substitution Rule; \textbf{\emph{Indefinite integrals}}& 2.0 \cr
	
	& 7.2 & Integration by Parts and Practicing Integration; \textbf{\emph{Indefinite integrals}} & 2.0 \cr
	
	& 7.3 & Rational Functions and Partial Fractions; \textbf{\emph{Indefinite integrals}} & 1.0 \cr
	
	& 7.4.1 & Improper Integrals (Unbounded Intervals);  \textbf{\emph{Definite integrals}} & 2.0\cr
	
	Chapter & 8 & Differential Equations  &(7 hours) \cr
	
	& 8.1 & Solving Differential Equations; \textbf{\emph{Differential equations}} & 4.0\cr
	
	& 8.2 & Equilibria and Their Stability; \textbf{\emph{Differential equations}} & 3.0\cr
	
	Chapter & 12 &  Probability and Statistics & (15 hours) \cr
	
	& 12.1 & Counting; \textbf{\emph{Probability}} & 3.0\cr
	
	& 12.2 & What Is Probability; \textbf{\emph{Probability}} & 3.0\cr
	
	& 12.3 & Conditional Probability and Independence; \textbf{\emph{Probability}} & 3.0\cr
	
	& 12.4 & Discrete Random Variables and Discrete Distributions; \textbf{\emph{Probability}} & 3.0\cr
	
	& 12.5 & Continuous Distributions; \textbf{\emph{Probability}} & 3.0\cr
	
	Chapter	& 9 & Linear Algebra and Analytic Geometry & (3 hours)\cr
	
	& 9.4 & Analytic Geometry; \textbf{\emph{Vectors}} & 3.0\cr
	
	&   & Total Number of Lecture Hours & 41.0\cr}
\vspace{.1in}
The instructor reserves the right to change the content of the course material if he/she perceives a need due to postponement of class caused by inclement weather, instructor illness, etc., or due to the pace of the course. Students are responsible for any announcement made in class. If you have to send an email, email with clear subject, course and section number, most importantly, do not reply to one of the earlier emails. If you have any issues talk to your instructor during office hours or before or after the class, do not completely rely on emails.
\end{document}
