\documentclass[11pt]{article}
\usepackage{lmodern}
\usepackage{amssymb,amsmath}
\usepackage{ifxetex,ifluatex}
\usepackage{fixltx2e} % provides \textsubscript
\ifnum 0\ifxetex 1\fi\ifluatex 1\fi=0 % if pdftex
  \usepackage[T1]{fontenc}
  \usepackage[utf8]{inputenc}
\else % if luatex or xelatex
  \ifxetex
    \usepackage{mathspec}
  \else
    \usepackage{fontspec}
  \fi
  \defaultfontfeatures{Ligatures=TeX,Scale=MatchLowercase}
\fi
% use upquote if available, for straight quotes in verbatim environments
\IfFileExists{upquote.sty}{\usepackage{upquote}}{}
% use microtype if available
\IfFileExists{microtype.sty}{%
\usepackage{microtype}
\UseMicrotypeSet[protrusion]{basicmath} % disable protrusion for tt fonts
}{}
\usepackage[unicode=true]{hyperref}
\hypersetup{
            pdfborder={0 0 0},
            breaklinks=true}
\urlstyle{same}  % don't use monospace font for urls
\usepackage{longtable,booktabs}
\usepackage{graphicx,grffile}
\makeatletter
\def\maxwidth{\ifdim\Gin@nat@width>\linewidth\linewidth\else\Gin@nat@width\fi}
\def\maxheight{\ifdim\Gin@nat@height>\textheight\textheight\else\Gin@nat@height\fi}
\makeatother
% Scale images if necessary, so that they will not overflow the page
% margins by default, and it is still possible to overwrite the defaults
% using explicit options in \includegraphics[width, height, ...]{}
\setkeys{Gin}{width=\maxwidth,height=\maxheight,keepaspectratio}
\IfFileExists{parskip.sty}{%
\usepackage{parskip}
}{% else
\setlength{\parindent}{0pt}
\setlength{\parskip}{6pt plus 2pt minus 1pt}
}
\setlength{\emergencystretch}{3em}  % prevent overfull lines
\providecommand{\tightlist}{%
  \setlength{\itemsep}{0pt}\setlength{\parskip}{0pt}}
\setcounter{secnumdepth}{0}
% Redefines (sub)paragraphs to behave more like sections
\ifx\paragraph\undefined\else
\let\oldparagraph\paragraph
\renewcommand{\paragraph}[1]{\oldparagraph{#1}\mbox{}}
\fi
\ifx\subparagraph\undefined\else
\let\oldsubparagraph\subparagraph
\renewcommand{\subparagraph}[1]{\oldsubparagraph{#1}\mbox{}}
\fi

\date{}

\begin{document}

\textbf{Purpose:} Determine the basics of how a circuit works and
discover the relationships between circuit elements.

\textbf{Directions:} Google
\href{https://phet.colorado.edu/en/simulations/circuit-construction-kit-dc}{\emph{``PHET
Circuit Construction Kit: DC''}}

Answer each of the following questions in as much detail as possible.
Start in the ``Intro'' section of the simulator.

\textbf{\emph{Part 1: Exploration}}

\begin{enumerate}
\def\labelenumi{\Alph{enumi}.}
\item
  For each of the circuit elements that can be found on the left-hand
  side of the simulator, describe them and explain what you think each
  of them are for.
\end{enumerate}

\begin{longtable}[]{@{}lll@{}}
\toprule
\textbf{Circuit Element} & \textbf{Description} &
\textbf{Purpose}\tabularnewline
\midrule
\endhead
Wire & &\tabularnewline
Battery & &\tabularnewline
Light Bulb & &\tabularnewline
Switch & &\tabularnewline
Fuse & &\tabularnewline
Resistor & &\tabularnewline
\bottomrule
\end{longtable}

\begin{enumerate}
\def\labelenumi{\Alph{enumi}.}
\item
  Underneath the lefthand panel, select the symbol that looks like this:
\end{enumerate}

Draw in the symbol for each circuit element.

\begin{longtable}[]{@{}llllll@{}}
\toprule
\textbf{Wire} & \textbf{Battery} & \textbf{Light Bulb} & \textbf{Switch}
& \textbf{Fuse} & \textbf{Resistor}\tabularnewline
\midrule
\endhead
& & & & &\tabularnewline
\bottomrule
\end{longtable}

\begin{enumerate}
\def\labelenumi{\roman{enumi}.}
\item
  What do you notice about the symbols for the other items (coin, dog,
  hand, etc.), and why do you think this is the case?
\end{enumerate}

\textbf{\emph{Part 2: Building a Circuit}}

Switch the circuit elements back to lifelike representations by clicking
this button:

\begin{enumerate}
\def\labelenumi{\Alph{enumi}.}
\item
  What do you think the blue circles represent?
\item
  Select a wire. What can you change about the wire?
\item
  Select a battery. What can you change about the battery?
\item
  Select a light bulb. What can you change about the light bulb?
\item
  Select a resistor. What can you change about the resistor?
\item
  Drag and drop elements onto the workspace and connect them together to
  make a working circuit. How do you know that the circuit is working?
\item
  What conditions must be true for the electrons to move?
\item
  Draw a schematic of your working circuit below:
\end{enumerate}

\begin{longtable}[]{@{}l@{}}
\toprule
\bottomrule
\end{longtable}

\begin{enumerate}
\def\labelenumi{\Alph{enumi}.}
\item
  What could you change about your circuit and still get it to work?
\end{enumerate}

\textbf{\emph{Part 3: Identifying Relationships}}

Create a circuit like the one shown below:

%\includegraphics[width=3.80833in,height=2.12500in]{media/image3.png}

\begin{enumerate}
\def\labelenumi{\Alph{enumi}.}
\item
  Which direction do the blue circles flow around the circuit?
\item
  Why do they flow in that direction?
\item
  \textbf{Prediction} - How will changing the resistance affect their
  flow?
\item
  \textbf{Prediction -} How will changing the voltage affect their flow?
\item
  Based on your observations, what do you think each of the following
  measures

  \begin{enumerate}
  \def\labelenumii{\roman{enumii}.}
  \item
    Current --
  \item
    Voltage --
  \item
    Resistance --
  \end{enumerate}
\end{enumerate}

Using the ammeter record how a change in voltage or resistance affects
the current flowing through the circuit. Then graph the relationship.
For each data set, choose a fixed value for your constant variable and
record it. You can also record your data in an excel sheet and draw a
graph on there, or use python.

\emph{Data Set 1 -- Voltage vs. Current}

Constant Variable: Resistance = \_\_\_\_\_\_\_\_\_\_

\begin{longtable}[]{@{}ll@{}}
\toprule
\textbf{Voltage}

Units:\_\_\_\_\_\_\_\_ & \textbf{Current}

Units:\_\_\_\_\_\_\_\_\tabularnewline
\midrule
\endhead
&\tabularnewline
&\tabularnewline
&\tabularnewline
&\tabularnewline
&\tabularnewline
&\tabularnewline
&\tabularnewline
\bottomrule
\end{longtable}

\emph{Graph}:

Describe the patterns and relationships that you see in your data table
and graph.

\emph{Data Set 2 -- Resistance vs. Current}

Constant Variable: Voltage = \_\_\_\_\_\_\_\_\_\_

\begin{longtable}[]{@{}ll@{}}
\toprule
\textbf{Resistance}

Units: \_\_\_\_\_\_\_\_ & \textbf{Current}

Units: \_\_\_\_\_\_\_\_\tabularnewline
\midrule
\endhead
&\tabularnewline
&\tabularnewline
&\tabularnewline
&\tabularnewline
&\tabularnewline
&\tabularnewline
&\tabularnewline
\bottomrule
\end{longtable}

\emph{Graph}:

Describe the patterns and relationships that you see in your data table
and graph.

\end{document}
