\documentclass[12pt,addpoints]{exam}
\usepackage{enumitem}
\usepackage{amsfonts,amssymb,amsmath, amsthm}
\usepackage{graphicx}
\usepackage{systeme}
\usepackage{pgf,tikz,pgfplots}
\pgfplotsset{compat=1.15}
\usepgfplotslibrary{fillbetween}
\usepackage{mathrsfs}
\usetikzlibrary{arrows}
\usetikzlibrary{calc}
\pagestyle{headandfoot}
\firstpageheadrule
\runningheader{Grade 10 Chapter 4}{}{Page \thepage\ of \numpages}
\runningheadrule
\author{St John Baptist De La Salle Catholic School, Addis Ababa}
\usepackage{geometry}
\geometry{
	a4paper,
	total={170mm,257mm},
	left=15mm,
	right=15mm,
	bottom=20mm,
	top=15mm,
}
\firstpagefooter{}{}{}
\runningfooter{}{}{}
\date{22/23 Academic Year}

\begin{document}
	\title{Grade 10 Chapter 4 Workbook Questions}
	\maketitle
	
	\begin{center}
		\subsection*{Questions}
	\end{center}
	
	
	\begin{questions}
		\question Discuss how the Hall effect could be used to obtain information on free charge density in a conductor.
		\question Explain why the magnetic field would not be unique (that is, not have a single value) at a point in space where magnetic
		field lines might cross. 
		\question List the ways in which magnetic field lines and electric field lines are similar and different.
		\question The force per meter between the two wires of a jumper cable being used to start a stalled car is 0.225 N/m. 
		\begin{enumerate}[label=(\roman*)]
			\item What is the current in the wires, given they are separated by 2.00 cm?
			\item Is the force attractive or repulsive(\textit{the current carrying wires in a jumper cable run in opposite directions})?
		\end{enumerate}
		\question To what direction should an electron be shot so that when it is put in a magnetic field in the direction of the negative Z-axis, the force acting on it is in the positive X-axis?
		\question A cosmic ray electron moves at $ 7.50 \times 10^6 \;\textbf{m/s}$ perpendicular to the Earth’s magnetic field at an altitude where field strength is $1.00 \times 10^{-5} \;\textbf{T} $. What is the radius of the circular path the electron follows?
		\question Calculate the inductive time constant of a circuit which has an inductor with an inductance of 6mH and a resistor of resistance 300$\varOmega$. If the EMF supplied by the battery is 60V, calculate the time needed for the current to drop to 0.1A. 
		\question Find the magnetic force(both the magnitude and direction) acting on a proton if its velocity is V=$1.6\times10^6\hat{\boldsymbol{j}}$ m/s and it is in a magnetic field of B=$2\hat{\boldsymbol{i}} + 8\hat{\boldsymbol{j}} + 72\hat{\boldsymbol{k}}$T
		\question Show that V/H and A/s are the same by doing a dimensional analysis.
		\question A long solenoid has 1000 turns uniformly distributed over a length of 0.40m. What current is required in the windings to produce a magnetic field of $\pi\times10^{-2}$G at the center of the solenoid?
		\question How can we decrease the effect of eddy currents?
		\question A 15.0 cm long rod moves at 6.00 m/s perpendicular to a magnetic field. What is the strength of the magnetic field if a 95.0 V emf is induced?
		\question Calculate the magnetic field strength needed on a 100-turn square loop 18.0 cm on a side to create a maximum torque of
		500Nm if the loop is carrying 25.0 A.
		\question What is the force and torque on a square-shaped 6A current carrying loop of conducting wire that has an area of 0.0064$m^2$ and surrounded by a permanent magnet with a field strength of B = $3.0\times10^5$T that is tilted at 30$^0$ to the loop?
		\question If a charged particle moves in a straight line through some region of space, can you say that the magnetic field in that region is necessarily zero?
		\question What is the angle between the current carrying wire and the magnetic field when the force exerted on the wire is half of the maximum force possible?
		\question Find the charge to mass ratio of a charge moving if it is moving at a speed of $v = 5.0\times10^3 m/s$ in a magnetic field of 0.08G and it has the same trajectory as an electron in the same magnetic field.
		\question What is the magnetic field 2cm away due to a straight current carrying wire made of Manganese if the wire has a volume $27cm^3$ and length 3cm, if it is switched on for 5 seconds?
		\question A uniform magnetic field of magnitude 1.2 T is directed along the negative y - axis. An electron moving at a speed of 0.2$c$ makes an angle of 60$^0$ with the y - axis. Answer the following questions.
		\begin{enumerate}[label=(\roman*)]
			\item What is the expected trajectory of the electron?
			\item Calculate the radius \& pitch of the trajectory.
		\end{enumerate}
		\question What are ferromagnetic materials? What are some common ferromagnetic materials?
		\question We have seen that charges and magnets have similar properties. How is a charge similar to and different from a magnet?
		\question If a current carrying wire of 2A and length 3m is in a region of space that is perpendicular to a magnetic field of 3T, what is the maximum force on the wire? What about the minimum force?
		\question An electron and a proton are shot with the same speed of $3\times10^7m/s$ perpendicular to the magnetic field in question 3. What are the magnitudes of the forces on the electron and the proton? What about the radius of the trajectories by the electron and the proton?
		\question A current carrying wire that is carrying a current of 4A is going into the page. What is the magnetic field strength due to the wire 2cm vertically above the wire?(Both B and $\hat{B}$)
		\question For two wires both going out of the page, show that the two wires attract.
	\end{questions}		
\end{document}