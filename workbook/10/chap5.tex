\documentclass[12pt,addpoints]{exam}
\usepackage{enumitem}
\usepackage{amsfonts,amssymb,amsmath, amsthm}
\usepackage{graphicx}
\usepackage{systeme}
\usepackage{pgf,tikz,pgfplots}
\pgfplotsset{compat=1.15}
\usepgfplotslibrary{fillbetween}
\usepackage{mathrsfs}
\usetikzlibrary{arrows}
\usetikzlibrary{calc}
\pagestyle{headandfoot}
\firstpageheadrule
\runningheader{Grade 10 Chapter 5}{}{Page \thepage\ of \numpages}
\runningheadrule
\author{St John Baptist De La Salle Catholic School, Addis Ababa}
\usepackage{geometry}
\geometry{
	a4paper,
	total={170mm,257mm},
	left=15mm,
	right=15mm,
	bottom=20mm,
	top=15mm,
}
\firstpagefooter{}{}{}
\runningfooter{}{}{}
\date{22/23 Academic Year}
\begin{document}
	\title{Grade 10 Chapter 5 Workbook Questions}
	\maketitle
	
	\begin{center}
		\subsection*{Questions}
	\end{center}
	\begin{questions}
	\question Let's say we want to study a signal visually. For an AC signal of a maximum voltage 12V and frequency of 60 Hz, draw a visual representation of what we would expect to see on an oscilloscope. You are free to give the oscilloscope the time base and gain control of your choice.
	\question State the uses of a transistor and explain how amplification is possible through a double junction. Draw the paths of current in the transistor and state Kirchhoff's Law. Explain why the word "amplification" is misleading while using it for transistors.
	\question Explain the difference between P-type and N-type semiconductors and how they are made. Explain how their properties gives rise to rectification. Explain what rectification is and state what the opposite process of rectification is(inversion) and state how we can rectify or invert current.
	\question Define and explain the following terms:
	\begin{enumerate}[label=(\roman*)]
		\item  Doping and impurities.
		\item  Acceptor and Donor atoms.
		\item  Conduction band theory and lattice structures.
	\end{enumerate}
	\question Explain what we mean by 0 and 1 electrical signals. Explain what logic gates are. Give 3 examples of ligc gates used in real life.
	\question In what way can we achieve a full wave rectification?
	\question An input of direct current is sent into an unknown electrical device and when current emerges out of the device, the output current is alternating. What device could the unknown be?
	\question What is the emission of conduction electrons from the hot meta in a Fermi valve is known as?
	\question Why is the Fermi valve referred to as a "valve"?
	\question Check whether the logic gates given below are equivalent or not.
	\begin{choices}
		\choice An \textbf{AND} and a \textbf{NOT-NAND}
		\choice An \textbf{OR} and a \textbf{NOT}
		\choice An \textbf{AND} and a \textbf{NAND}
		\choice A \textbf{NOT} and an \textbf{XOR}
	\end{choices}
	\question What does it mean when a P-N junction is forward biased? What about when it is reverse biased?
	\question Why is rectification by a single diode always half-wave?
	\question Plot a signal for a CRO measuring a signal of frequency 200Hz and maximum voltage 8V if the gain control is 4V/cm and time base is 2ms/cm.
	\question The collector current of a transistor is 4.2 A for a base current of 3.4 mA. What is the current gain?
	\question The base current of a transistor is 5.4 A, and its current gain is 1200. What is the collector current?
	\end{questions}		
\end{document}