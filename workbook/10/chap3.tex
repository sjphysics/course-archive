\documentclass[12pt,addpoints]{exam}
\usepackage{enumitem}
\usepackage{amsfonts,amssymb,amsmath, amsthm}
\usepackage{graphicx}
\usepackage{systeme}
\usepackage{pgf,tikz,pgfplots}
\pgfplotsset{compat=1.15}
\usepgfplotslibrary{fillbetween}
\usepackage{mathrsfs}
\usetikzlibrary{arrows}
\usetikzlibrary{calc}
\pagestyle{headandfoot}
\firstpageheadrule
\runningheader{Grade 10 Chapter 3}{}{Page \thepage\ of \numpages}
\runningheadrule
\author{St John Baptist De La Salle Catholic School, Addis Ababa}
\usepackage{geometry}
\geometry{
	a4paper,
	total={170mm,257mm},
	left=15mm,
	right=15mm,
	bottom=20mm,
	top=15mm,
}
\firstpagefooter{}{}{}
\runningfooter{}{}{}
\date{22/23 Academic Year}

\begin{document}
	\title{Grade 10 Chapter 3 Workbook Questions}
	\maketitle
	
	\begin{center}
		\subsection*{Questions}
	\end{center}
	
	
	\begin{questions}
		\question Define current and explain current in different ways. For example, state why although capacitors can be treated as open switches, current still runs in the circuit.
		\question A conducting copper wire has a diameter of 2.228 mm. What magnitude current flows when the drift velocity is 1.00
		mm/s?
		\question Given that the density of Manganese is $3.7g/cm^3$ and that we assume that there are 3 mobile electrons per each atom, calculate the electron density of a conducting wire made of Manganese.
		\question Power outages are common in Ethiopia and hence rechargeable batteries are common. One such example of battery is a "power bank" that we can use to charge our devices. Aaron's "power bank" boasts a $6000mAh$ capability. What physical quantity does $mAh$ represent?
		\question Why are two conducting paths from a voltage source to an electrical device needed to operate the device?
		\question Why isn’t a bird sitting on a high-voltage power line electrocuted? What happens when it steps its feet on both wires?
		\question Discuss both the macroscopic and microscopic aspects of Ohm's Law.
		\question What is the effective resistance of a car’s starter motor when 200 A flows through it as the car battery applies 12.0 V to the motor?
		\question Find the conductivity and resistivity of a material if it is 50.0 m long with a 0.050 mm diameter and has a resistance of 80$\Omega$ at 20$^0$C?
		\question What does ammeter measure? How should it be connected to the circuit? Why? What about a voltmeter?
		\question If there are $n$ identical resistors of resistance R in a network and 40\% of them are connected in series while the other 60\% are connected in parallel, find the effective resistance in terms of n \& R. 
		\question Given a battery, an assortment of resistors, and a variety of voltage and current measuring devices, describe how you
		would determine the internal resistance of the battery.
		\question The hot resistance of a flashlight bulb is $5\Omega$, and it is run by a 2.8-V alkaline cell having a internal
		resistance of $0.3\Omega$.
		\begin{enumerate}[label=(\roman*)]
			\item What current flows through the bulb?
			\item Calculate the power dissipated by the bulb.
			\item What is the efficiency of the bulb?
		\end{enumerate}
		\question Show that for two resistors $R_1$ and $R_2$, the effective resistance when they are combined is larger when the resistors are in series than when they are in parallel.
	\end{questions}		
\end{document}