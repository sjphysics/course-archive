\documentclass[12pt,addpoints]{exam}
\usepackage{enumitem}
\usepackage{amsfonts,amssymb,amsmath, amsthm}
\usepackage{graphicx}
\usepackage{systeme}
\usepackage{pgf,tikz,pgfplots}
\pgfplotsset{compat=1.15}
\usepgfplotslibrary{fillbetween}
\usepackage{mathrsfs}
\usetikzlibrary{arrows}
\usetikzlibrary{calc}
\pagestyle{headandfoot}
\firstpageheadrule
\runningheader{Grade 10 Chapter 1}{}{Page \thepage\ of \numpages}
\runningheadrule
\author{St John Baptist De La Salle Catholic School, Addis Ababa}
\usepackage{geometry}
\geometry{
	a4paper,
	total={170mm,257mm},
	left=15mm,
	right=15mm,
	bottom=20mm,
	top=15mm,
}
\firstpagefooter{}{}{}
\runningfooter{}{}{}
\date{22/23 Academic Year}

\begin{document}
	\title{Grade 10 Chapter 1 Workbook Questions}
	\maketitle
	
	\begin{center}
		\subsection*{Questions}
	\end{center}
	
	
	\begin{questions}
		\question Define the following terms and explain what relationships they have with respects to projectile motion.
		\begin{enumerate}[label=(\roman*)]
			\item Trajectory
			\item Projectile
			\item Air resistance
			\item Kinematics
			\item Drag
		\end{enumerate}
		\question During a fireworks display, a shell is shot into the air with an initial speed of $64.0 m/s$ at an angle of $65.0^0$ above the horizontal. The fuse is timed to ignite the shell just as it reaches its highest point above the ground.
		\begin{enumerate}[label=(\roman*)]
			\item  Calculate the vertical distance above the ground at which the shell explodes.
			\item  How much time passed between the launch of the shell and the explosion?
			\item  What is the horizontal displacement of the shell when it explodes?
			\item  If the shell didn't explode, calculate the velocity it will have just before touching the ground.
		\end{enumerate}
		\question Suppose a large rock is ejected during a tectonic activity with a speed of $35.0 m/s$ and at an angle $38^0$ above the horizontal. The rock strikes a plateau at an altitude 400.0 m lower than its starting point. 
		\begin{enumerate}[label=(\roman*)]
			\item Calculate the time it takes the rock to follow this path.
			\item What are the magnitude and direction of the rock’s velocity at impact.
			\item What is the highest vertical range of distance it achieves during its trajectory?
		\end{enumerate} 
		\question A person standing on the edge of the rooftop of a skyscraper accidentally throws his phone straight up with an initial velocity of $20.0 m/s$. The rock misses the edge as it falls back to earth. Calculate the position and velocity of the rock 1.00 s, 2.00 s, and 3.00 s after it is thrown, neglecting the effects of air resistance.
		\question Helicopters have a small propeller on their tail to keep them from rotating in the opposite direction of their main lifting blades. Explain in terms of Newton’s third law why the helicopter body rotates in the opposite direction to the blades.
		\question An ultracentrifuge accelerates from rest to 80,000 rpm in 4.00 min.
		\begin{enumerate}[label=(\roman*)]
			\item What is its angular acceleration in $rad/s^2$?
			\item What is the tangential acceleration of a point 6.50 cm away from the axis of rotation?
			\item What is the radial acceleration in $m/s^2$ and multiples of $g$ of the point in (ii) at full rpm?
		\end{enumerate}
		\question Veronica exerts a force of 180 N tangential to a 0.4-m radius 60.0-kg grindstone that is geometrically a solid disk.
		\begin{enumerate}[label=(\roman*)]
			\item What torque is veronica exerting on the grindstone?
			\item If the grindstone generates no opposing friction, what is its rotational acceleration?
			\item What is the angular acceleration if there is an opposing frictional force of 18.0 N exerted 2 cm from the axis?
		\end{enumerate}
		\question Calculate the angular momentum of a ballerina spinning at 7.00 rev/s given her moment of inertia is 6$kg-m^2$.
		\begin{enumerate}[label=(\roman*)]
			\item She reduces her spin rate by extending her arms and increasing her moment of inertia. Find the value of her moment of inertia if her angular velocity decreases to 3 rev/s.
			\item Suppose instead she keeps her arms in and allows friction of the ice to slow her to 2.00 rev/s. What average torque
			was exerted if this takes 15.0 s?
		\end{enumerate}
		\question Given that the mass of the Earth is around $5.96\times10^{24}$ kg, answer the following questions.
		\begin{enumerate}[label=(\roman*)]
			\item Calculate the angular momentum of Earth on its axis.
			\item What is the angular momentum of Earth in its orbit around the Sun?
			\item Calculate the rotational kinetic energy of Earth on its axis.
			\item What is the rotational kinetic energy of Earth in its orbit around the Sun?
		\end{enumerate}
		\question While getting ready for a free-kick, Addis rotates his leg about the hip joint. The moment of inertia of the leg is 2.8$kg-m^2$ and its rotational kinetic energy is 175 J.
		\begin{enumerate}[label=(\roman*)]
			\item What is the angular velocity of the leg?
			\item What is the velocity of tip of Addis’s shoe if it is 1.2 m from the hip joint?
			\item Explain how the football can be given a velocity greater than the tip of the shoe (necessary for a decent kick
			distance). (\textit{Hint: think of the effect time has while kicking a ball.})
		\end{enumerate}	
		\question The Moon and Earth rotate about their common center of mass, which is located about 4700 km from the center of Earth.
		(This is 1690 km below the surface of the Earth.)
		\begin{enumerate}[label=(\roman*)]
			\item Calculate the magnitude of the acceleration due to the Moon’s gravity at that point.
			\item For a hypothetical asteroid of mass 9000 kg, estimate the force exerted on it by  the moon if it is at that point.
		\end{enumerate}
		\question State three of Kepler's laws and the physical reasoning behind each law.
		\question We know that the Moon orbits Earth each 27.3 days and that it is an average distance of $3.84\times10^{8}$m from the center of Earth. 
		\begin{enumerate}[label=(\roman*)]
			\item A geosynchronous Earth satellite is one that has an orbital period of precisely 1 day. Such orbits are useful for
			communication and weather observation because the satellite remains above the same point on Earth (provided it orbits in the
			equatorial plane in the same direction as Earth’s rotation). Calculate the radius of such an orbit.
			\item Calculate the period of an artificial satellite orbiting at an average altitude of 2000 km above Earth’s surface.
		\end{enumerate}
	\end{questions}		
\end{document}