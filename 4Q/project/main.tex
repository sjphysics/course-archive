\documentclass[11pt,addpoints]{exam}
\usepackage{enumitem}
\usepackage{amsfonts,amssymb,amsmath, amsthm}
\usepackage{graphicx}
\usepackage{systeme}
\usepackage{pgf,tikz,pgfplots}
\pgfplotsset{compat=1.15}
\usepgfplotslibrary{fillbetween}
\usepackage{mathrsfs}
\usetikzlibrary{arrows}
\usetikzlibrary{calc}
\pagestyle{headandfoot}
%\firstpageheadrule
\runningheader{Homework 2}{}{Page \thepage\ of \numpages}
\runningheadrule
\author{Aaron GK}
\usepackage{geometry}
\geometry{
	a4paper,
	total={170mm,257mm},
	left=15mm,
	right=15mm,
	bottom=20mm,
	top=15mm,
}
\firstpagefooter{}{}{}
\runningfooter{}{}{}


\begin{document}
	\title{St John Baptist De La Salle Catholic School, Addis Ababa\\
		\large Physics Projects \\
		4th Quarter}
	\maketitle
	\begin{center}
		\subsubsection*{Information}
		\begin{itemize}
			\item Both projects are due on \textbf{Monday}, \textbf{June 12th}.
			\item You should work on the projects\textbf{in groups} and consult me if you have any questions.
			\item For simplicity of grading and collaboration, you should do the projects in two teams of \textbf{1-to-5} groups of your choice for project 1. For project 2, use the selected groups in the project 2 section.
		\end{itemize}
	\end{center}
	\begin{center}
		\subsection*{Project 1}
	\end{center}
	\begin{questions}
		\question Take any topic of your choice beginning from Electrodynamics up to where we are right now( includes topics in current, magnetic fields, motor effect, dynamo effect, EM waves \& the like), then prepare an artistic explanation of the topic you chose. The project you submit could have many different formats and you are free to choose from ones below or come up with anything you like \& think would be a great delivery.
		\begin{itemize}
			\item A parody of a song that could have lyrics with the topics you chose
			\item A TikTok/YouTube video that is informative of the topics you chose, but at the same time trendy \& fun.
			\item A video of your topic of choice based on some famous act or a news parody.
			\item You could also prepare a vlog/v-cast/website on the topic of your choice.
			\item Or come up with something creative and interesting way to present the physics topics of your choice.
		\end{itemize}
		\textbf{NOTE}; You are not required to post the assignments publicly, but you can do so if you think your work is inspiring to the world. As for the submission, you can submit your work by sharing it through Google Drive at \textbf{aaronkebede@stjohn.edu.et}.
		\begin{center}
			\subsection*{Project 2}
		\end{center}
		\question For this project, we'll get very handsy. There will be a set of materials that you will be making, but at the same time, you also need to prepare a manual/document explaining how your material operates and the physics concept behind it.
		\begin{itemize}
			\item A Simple Motor (\textbf{Groups 1 \& 10})
			\item A Simple Manual Generator (\textbf{Groups 6 \& 7})
			\item A Periscope (\textbf{Groups 4 \& 9})
			\item A Simple Compass Needle and an Electromagnet (\textbf{Groups 3 \& 8})
			\item A Pinhole Camera \& A schematic diagram of reflection by spherical mirrors (\textbf{Groups 2 \& 5})
		\end{itemize}
		\subsubsection*{A few things to keep in mind}
		\begin{enumerate}
			\item If -for any reason- working on the actual models is not possible, you can contact me and we can discuss ways to mitigate the issues. There is always the option of opting in to a simulation made by you about the topic you have been given.
			\item For an 11th(\& 12th) group, members are welcome to join any of the groups that they would like. 
			\item Extra credits will be provided on the basis of exceptional quality of work, and on work that is exceptionally beyond what was asked. 	
			\item Although both projects are group projects, individual participation in each group will be valued highly and will be graded.
		\end{enumerate}
		
			
	\end{questions}		
\end{document}