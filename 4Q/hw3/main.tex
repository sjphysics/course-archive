\documentclass[11pt,addpoints]{exam}
\usepackage{enumitem}
\usepackage{amsfonts,amssymb,amsmath, amsthm}
\usepackage{graphicx}
\usepackage{systeme}
\usepackage{pgf,tikz,pgfplots}
\pgfplotsset{compat=1.15}
\usepgfplotslibrary{fillbetween}
\usepackage{mathrsfs}
\usetikzlibrary{arrows}
\usetikzlibrary{calc}
\pagestyle{headandfoot}
%\firstpageheadrule
\runningheader{Homework 2}{}{Page \thepage\ of \numpages}
\runningheadrule
\author{Aaron GK}
\usepackage{geometry}
\geometry{
	a4paper,
	total={170mm,257mm},
	left=15mm,
	right=15mm,
	bottom=20mm,
	top=15mm,
}
\firstpagefooter{}{}{}
\runningfooter{}{}{}


\begin{document}
	\title{St John Baptist De La Salle Catholic School, Addis Ababa\\
		\large Homework 3 \\
		4th Quarter}
	\maketitle
	\begin{center}
		\fbox{\fbox{\parbox{6in}{\centering
					Notes, and use of other aids is allowed.  Read all directions carefully and write your answers in the space provided.  To receive full credit, you must show all of your work. \textbf{Cheating or indications of cheating and similar answers will be punished accordingly}. 
		}}}
		\subsubsection*{Information}
		\begin{itemize}
			\item The homework is due on \textbf{Monday}, \textbf{June 12th}.
			\item You should Work on it \textbf{in groups} and consult me if you have any questions. As I have reiterated multiple times, cheating between groups will have a serious consequence.
			\item For purposes of neatness and simplicity of grading, you should do the homework on an \textbf{A-4 paper}.
		\end{itemize}
	\end{center}
	\begin{center}
		\subsection*{Questions on Image formation by lenses}
	\end{center}
	
	\begin{questions}
		\question Explain issues of near-sightedness and far-sightedness pictorial and explain how a solution can be provided to both.
		\question Using ray tracing, show the image created when an object is placed at the following places on a converging lens:
		\begin{itemize}
			\item beyond 2F
			\item at 2F
			\item between 2F \& F
			\item at F
			\item between the lens \& F
		\end{itemize}
		\question In what position(s), will a converging lens of focal length 8cm form images of an object on a screen located 40cm from the object?
		\question Assume the projector we use in class has a 100-D lens. If we are observing an upright image of dimension 1.2m x 1.8m 3m away, what dimensions must have the smaller slides in the projector have been?
		\question A camera lens used for taking close-up photographs has a focal length of 0.3 cm. The farthest it can be placed from the film is 50.0 mm.
		\begin{itemize}
			\item What is the closest object that can be photographed?
			\item What is the magnification of this closest object?
		\end{itemize}
		
		
		
		
		\begin{itemize}
			\item How far away must the screen be to produce a sharp image if the lens is 20cm from the slide?
			\item If the slide has dimensions of 2.0mm by 4.0mm, what are the dimensions of the image?
		\end{itemize}
		\subsection*{Advanced Problems}
		\question A glass($n=1.50$) lens has a focal length of 15cm in air, what is its focal length in water? If the radius of one-half of the lens is 10cm, what is the radius of the other half?
		\question For two lenses that are in contact, show that the combination of their focal lengths($f$) can  be given as follows if one of the lenses has a focal length $f_1$ and the other has $f_2$, $\dfrac{1}{f}=\dfrac{1}{f_1}+\dfrac{1}{f_2}$
	\end{questions}		
\end{document}